% ----------------------------------------------------------
\chapter{Introdução}
\label{chp:LABEL_CHP_1}

% ----------------------------------------------------------

Segundo o \citeonline{Sebrae}, em abril de 2020, a pandemia do coronavírus afetou 98\% do setor de eventos, devido a cancelamentos, renegociações, entre outros motivos para evitar aglomerações. Destes eventos afetados, muitos empresários estão se preocupando com aprimorar sua gestão, e de acordo com o site feiras do Brasil \citeonline{FEIRA} o número de eventos online para 2021 já supera o número presencial devido à alta demanda. Além disso, de acordo com estudo da Associação Brasileira de Promotores de Eventos (Abrape), 51,9\% dos eventos previstos para ocorrer neste período foram cancelados ou adiados.

Para que estes eventos sejam realizados de uma forma mais eficiente, utilizam-se diversas ferramentas online que possibilitam o gerenciamento de eventos remotos e presenciais, entre essas:
\begin{itemize}
    \item \textbf{Even3: }O Even3 é uma plataforma online a qual disponibiliza espaço para que instituições e/ou organizações criem seus próprios eventos e os hospedem, neste também é possível a criação de eventos gratuitos e pagos, além da emissão de certificados. E dito isso, segundo o site do Reclame Aqui o maior problema deles é sua emissão de certificado, que apresenta problemas por parte da própria aplicação.
    \item \textbf{OCS:} O Open Conference System é um software que teve seu desenvolvimento paralisado pela equipe, e que possui seu código aberto onde o usuário pode elaborar o seu próprio site de compartilhamento de conferências acadêmicas. Todo este processo é realizado de forma manual, como a página do site por exemplo, mas há exceções do que vem fornecido pelo sistema, que é o cadastro de participantes, por exemplo.
    \item \textbf{Sympla: }O Sympla é uma plataforma, similar ao Even3, que se propõe a gestão completa e otimizada através de dezenas de ferramentas simples de usar, com eventos gratuitos e pagos também, além de shows e eventos. Seu maior problema, segundo a sua página no Reclame Aqui é no quesito de ingressos, principalmente em sua compra e recebimento.
    \item \textbf{EventBrite: }O Eventbrite é uma plataforma global de ingressos de autoatendimento para experiências ao vivo que permite a qualquer pessoa criar, compartilhar, encontrar e participar de eventos, festivais, maratonas e conferências, entre outros. Mas grande parte de seus problemas também se dão na compra e recebimento de ingressos, de acordo com reclamações de sua página no Reclame Aqui.
    \item \textbf{S-EVA: }O S-EVA (Sistema de Eventos Acadêmicos) é um protótipo que implementa a solução de eventos acadêmicos, no qual o seu foco é utilizar um banco de dados integrado a um framework para interface gráfica, além de avaliar trabalhos correlatos e destacar pontos importantes sobre gestão de eventos.
\end{itemize}

As quatro primeiras soluções são encontradas diretamente no mercado e estão prontas para o uso, porém o S-EVA foi feito como um protótipo em 2016 por meio de um projeto acadêmico.

O propósito deste trabalho de conclusão de curso (TCC) é criar um protótipo de aplicação web que gerencie eventos acadêmicos de forma que possua uma conceitos almejados no desenvolvimento de aplicações web atualmente: design responsivo, fluído, tendo um foco na User Interface e User Experiencie, além de demonstrar a divisão de responsabilidades entre as camadas de software por meio do padrão de desenvolvimento MVC e programação reativa.

% ----------------------------------------------------------
% JUSTIFICATIVA
% ----------------------------------------------------------

\section{Justificativa}

Aplicar os conceitos aprendidos teoricamente em cursos de tecnologia da informação  visando um design fluido, com foco na User Interface e User Experience. Além destes, a modelagem de banco de dados, e a implementação de uma metodologia de desenvolvimento e projeto para a aplicação proposta por meio de frameworks modernos.

% ----------------------------------------------------------
% OBJETIVOS
% ----------------------------------------------------------

\section{Objetivos}
O objetivo geral deste trabalho de conclusão de curso é desenvolver um protótipo de uma aplicação web para gerenciar eventos acadêmicos online, similar aos já citados, focado na simplicidade de uso, otimização de funções e regularidade de modelagem como forma de aplicar os conhecimentos adquiridos no curso relacionados a web, banco de dados e engenharia de software. FALTA ARRUMAR

\begin{itemize}
    \item Modelar a arquitetura do sistema.
    \item Implementação do modelo de banco de dados.
    \item Desenvolvimento da camada de backend em Laravel.
    \item Desenvolvimento da camada de frontend em Next.js.
    \item Implementação do funcionamento do gerenciador de eventos.
\end{itemize}

% ----------------------------------------------------------
% ESTRUTURA DO TRABALHO
% ----------------------------------------------------------
\section{Estrutura do trabalho}

Este trabalho está dividido em seis capítulos. O capítulo \ref{chp:LABEL_CHP_2} apresenta o referencial teórico sobre ... . O capítulo \ref{chp:LABEL_CHP_3} expõe a metodologia ....
No capítulo \ref{chp:LABEL_CHP_4} é retratado o desenvolvimento .... No capítulo \ref{chp:LABEL_CHP_5} são apresentados os resultados obtidos.... Por fim, no capítulo \ref{chp:LABEL_CHP_6} são feitas as
considerações finais e trabalhos futuros, seguidas pelas referências bibliográficas.