\chapter{Conclusão}\label{chp:LABEL_CHP_6}

Neste trabalho, foi desenvolvida e apresentada um protótipo de uma aplicação gerenciadora de eventos acadêmicos gratuitos online, que facilita a criação e a visualização para as instituições e para o participante.

O protótipo apresentado é capaz do gerenciamento de usuários, instituições e eventos, assim como a organização por três funções distintas de usuários. Somando essas características, há uma interface intuitiva que visa facilitar todos os processos para seus usuários, sejam apresentadores, participantes ou gerenciadores dos eventos. Há características similares aos apresentados na introdução, unindo características relevantes de cada um, e transformando-as num protótipo capaz de melhorar o gerenciamento dos eventos cadastrados.

A implantação deste sistema, inicialmente, pode ser feita em qualquer âmbito acadêmico, visto que abrange os mais variados tipos de eventos acadêmicos, podendo substituir, em caso de serem por meios digitais, o atualmente utilizado na instituição. Ou seja, a sua geração de certificados, a qual possibilita até mesmo a verificação e envio por e-mail, facilita todo o gerenciamento ao selecionar os participantes e apresentadores que de fato estiveram no evento.

% A utilização de novas tecnologias, como Next.js e Laravel, mostra que é possível criar uma aplicação moderna, com utilizações de boas práticas de programação, segundo \citeonline{BOASPRATICAS}, além de estarem em frameworks modernos e versionados. https://engsoftmoderna.info/
%Tal utilização, além de reforçar a segurança, melhoram o desenvolvimento do código, dado as facilidades que cada um apresenta, como por exemplo a criação de páginas estáticas e dinâmicas no Next.js e o Eloquent ORM do Laravel, que facilita as consultas no SGBD, assim como facilita a migração entre SGBDs diferentes, caso seja necessário.

Este trabalho poderá contribuir com as instituições que realizam recorrentemente tais eventos acadêmicos.  E, devido aos frameworks e tecnologias utilizados neste projeto, a aplicação funciona de maneira responsiva, podendo ser utilizada, tanto em computadores de uso pessoal, como em dispositivos móveis sem que haja maiores problemas, como a perda de funcionalidades e quebras no layout.


