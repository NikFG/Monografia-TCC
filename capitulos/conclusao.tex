\chapter{Conclusão}\label{chp:LABEL_CHP_6}

Neste trabalho, foi desenvolvido e apresentado um protótipo de uma aplicação gerenciadora de eventos acadêmicos gratuitos online, que facilita a criação e a visualização para os cinco perfis de usuários.

O protótipo apresentado é capaz do gerenciamento de usuários, instituições e eventos, assim como a organização por três funções distintas de usuários. Somando essas características, há uma interface intuitiva que visa facilitar todos os processos para seus usuários, sejam apresentadores, visitantes, participantes ou gerenciadores dos eventos (supervisores e associados). Há características similares aos sistemas apresentados na introdução, unindo características relevantes de cada um, e transformando-as num protótipo capaz de gerenciar os eventos cadastrados.

A implantação deste sistema, inicialmente, pode ser feita em qualquer âmbito acadêmico, visto que abrange os mais variados tipos de eventos acadêmicos de maneira remota, podendo substituir uma aplicação utilizada atualmente em uma determinada instituição. Ou seja, a sua geração de certificados, a qual possibilita até mesmo a verificação e envio por e-mail, facilita todo o gerenciamento ao selecionar os participantes e apresentadores que estiveram de fato no evento.

O método de adquirir ingressos poderia diferir, sendo mais similar a um \textit{ecommerce}, contendo carrinho com a possibilidade de selecionar diversas atividades de diversos eventos. Há também funcionalidades que poderiam ser agregadas ao sistema, tirando-o apenas do ramo acadêmico, como impressão de ingressos via qrcode. 

Este trabalho poderá contribuir com as instituições que realizam recorrentemente tais eventos acadêmicos. Devido aos \textit{frameworks} e tecnologias utilizados neste projeto, a aplicação funciona de maneira responsiva, podendo ser utilizada, tanto em computadores de uso pessoal, como em dispositivos móveis, sem que haja maiores problemas, como a perda de funcionalidades e quebras no layout.

Para trabalhos futuros, é necessário que tenha integração com APIs de \textit{gateway} de pagamentos para fornecer à aplicação funcionalidades de ingressos pagos, seja via cartão de crédito/débito ou pix, visto que nem sempre um evento pode ser algo gratuito. A aplicação poderá fazer uma reserva mediante a confirmação do pagamento, gerando um identificador único ao ingresso e podendo ser apresentado em formato de \textit{qrcode} ou o texto do código em si, que deve ser validado ao comparecer às atividades compradas. 

A solução adotada pode apresentar dificuldades ao escalar para instituições de vários campi por não atender ao requisito hierárquico. Como trabalho futuro, um painel administrativo poderá ser criado como meio de gerenciar por uma instituição centralizadora da informação, como exemplo: a matriz e filiais de uma instituição acadêmica, de modo que tenham até mais níveis de usuário para cada instituição ou de forma geral.

Alguns eventos hoje não são atualizados sem a necessidade de atualizar a página, o que pode causar números divergentes da realidade de participantes, por exemplo. Uma solução para trabalhos futuros pode ser a implementação por meio de \textit{websockets} que geram atualizações por meio de eventos no banco de dados e atualizam a página apenas no dado escolhido, neste exemplo o número de participantes.