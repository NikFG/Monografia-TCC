Para trabalhos futuros, é interessante que se agregue não apenas os eventos online, e sim os eventos presenciais, visto que com o fim da pandemia, a quantidade desses eventos pode aumentar consideravelmente em relação aos últimos dois anos. Tal implementação pode ser feita com utilização de um qr code e/ou via NFC em smartphones que possuam tal tecnologia, para validar a presença do participante e do apresentador. Outra possibilidade é expandir além dos eventos acadêmicos, podendo ser para qualquer evento, tendo como um exemplo um show de música, como o Rock in Rio, que realiza a venda de tais ingressos e possuem uma quantidade limitada de participantes (vale ressaltar que neste caso não se precisa necessariamente de um certificado de participação, apenas o ingresso). E, por fim, outra possibilidade é a migração da versão de dispositivos móveis para um aplicativo de fato, visto que toda estruturação do backend está pronta, bastaria apenas consumir a API com tecnologias preparadas exclusivamente para smartphones, por exemplo.