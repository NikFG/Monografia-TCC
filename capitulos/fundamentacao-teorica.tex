% ----------------------------------------------------------
\chapter{Fundamentação Teórica}\label{chp:LABEL_CHP_2}
% ----------------------------------------------------------

\section{Engenharia de Software}
A tendência é que softwares fiquem cada vez maiores e mais complexos, e isto se deve ao poder computacional crescente a cada ano. E isto faz com que os usuários criem maiores expectativas em relação aos sistemas de software, que não apenas devem transmitir informações pela internet de forma rápida e segura, mas também adaptados a necessidade de quem o usa. \cite{SOMMERVILE}.

A engenharia de software abrange os processos com um conjunto de métodos e diversas ferramentas que possibilitam aos profissionais desenvolverem sistemas de software com melhor planejamento \cite{PRESSMAN}. Tendo técnicas as quais irão se construir com base na especificação, projeto e evolução dos programas, que normalmente não são relevantes para o primeiro estágio de qualquer processo de projeto de software é o desenvolvimento de uma compreensão dos relacionamentos entre o software que está sendo projetado e seu ambiente externo. E isto é essencial para decidir qual funcionalidade o sistema terá e poderá ser apresentada ao usuário final, assim como a estruturação do próprio projeto e principalmente estabelecer os limites do sistema, ou seja, até que ponto será desenvolvido baseado no contexto do software \cite{SOMMERVILE}.

Para que isso seja planejado em uma aplicação e utilizada corretamente, de acordo com \cite[p.126]{SOMMERVILE}: \begin{quote}
    “Modelos de contexto do sistema e modelos de interação apresentam visões complementares dos relacionamentos entre um sistema e seu ambiente.”
\end{quote} 
\subsection{Caso de uso}
Segundo \citeonline{PRESSMAN}, a UML é uma linguagem padrão para descrever e documentar um projeto de software que reúne diversos grupos de notações de modelagem, sendo cada um com sua devida sintaxe e semântica predeterminadas. Cada um desses elementos pode ser extensível, permitindo, deste modo, que sejam adaptáveis às características específicas do projeto a ser desenvolvido. A UML independe de linguagem de programação, framework e dos processos de desenvolvimento do software, fazendo com que diferentes abordagens possam ser utilizadas durante o processo \cite{BEZERRA}.

O diagrama UML de caso de uso determina as funcionalidades e características presentes no software sendo uma visão geral do ponto de vista do usuário que descreve como este irá interagir com o sistema, seguindo determinadas ações para realizar um objetivo, como por exemplo fazer um login. Há uma padronização em seu desenvolvimento, representada pelas seguintes características, segundo \citeonline{PRESSMAN} e \citeonline{SOMMERVILE}:
\begin{itemize}
    \item \textbf{Ator: }representado por figuras de um boneco palito, é aquele que irá interagir com o sistema, sendo este externo ao sistema desenvolvido.
    \item \textbf{Caso de uso: }representado por uma elipse no diagrama, estes demonstram as ações que os atores podem realizar dentro do sistema, ou seja, identificam as interações individuais entre o sistema e seus usuários. 
    \item \textbf{Sistema: }elemento opcional representado por um retângulo para definir o objetivo de cada parte do diagrama de caso de uso, sendo 
\end{itemize}

\subsection{Processo unificado}
O processo unificado é um modelo incremental e iterativo derivado de trabalhos sobre a UML que ilustra as práticas na especificação, no projeto e prevê a prototipação aliada a entrega incremental do desenvolvimento. o PU é descrito em cima de três perspectivas, que podem ser descritas como dinâmica, que mostra as fases do modelo ao longo do tempo, estática, que mostra as atividades realizadas no processo e prática, que sugere boas práticas a serem usadas durante o processo \cite{SOMMERVILE}.

Segundo \citeonline{BEZERRA} existem cinco fases do PU, sendo estas:
\begin{itemize}
    \item \textbf{Fase de concepção: }identifica as entidades externas que irão interagir com o sistema, descrevendo os requisitos do projeto, que se desenvolve todo o planejamento para que seja possível as futuras iterações incrementais do processo. Nesta etapa é desenvolvido, também, a arquitetura provisória do sistema, que será refinada e expandida nos processos subsequentes.
    \item \textbf{Fase de elaboração: }refina e expande, por meio da ampliação da arquitetura do sistema, a fase de concepção, criando o modelo de requisitos e o modelo de implementação, mesmo ainda não oferecendo os recursos necessários para a utilização do sistema. Define-se as tecnologias que serão utilizadas ao longo do projeto, como frameworks e banco de dados, por exemplo.
    \item \textbf{Fase de construção: }é a fase que será desenvolvido o sistema por meio da programação, ou seja, as iterações desta etapa serão constituídas por fazer uma parte do sistema e a integrar as demais existentes, podendo serem desenvolvidas em paralelo. Tudo isto deve ser feito seguindo as documentações geradas nesta etapa e nas anteriores, para que não haja falha na entrega dos requisitos necessários. E ao fim desta fase, deve-se ter um sistema de software documentado e funcionando corretamente em seu ambiente operacional, de modo que já possa ser utilizado em modo de teste pelo usuário.
    \item \textbf{Fase de transição: }é a fase que entrega-se o software ao usuário final para testes, juntamente aos manuais de utilização e instalação, se for o caso, fazendo com que sejam relatados possíveis defeitos e ajustes necessários. Na conclusão desta etapa, o software poderá operar em modo de produção.
    \item \textbf{Fase de produção: }disponibiliza o ambiente operacional ao usuário, monitorando o seu uso contínuo, por meio de suporte ao ambiente operacional, realizando alterações e correções se forem necessários.
\end{itemize}

No PU, a iteração ocorre de duas formas: entre as cinco fases e entre a própria fase, fazendo com que possam acontecer de maneira concorrente e escalonada. Pode-se retornar a cada uma dessas etapas caso seja necessário complementar e/ou corrigir algum requisito ou definir uma nova integração, por exemplo, tornando-se incremental a cada etapa.


\section{Padores de projetop}

Padrões de projeto
Os padrões de projeto foram obtidos a partir das ideias apresentadas por Christopher Alexander (ALEXANDER
et al., 1977), que sugeriu haver padrões comuns de projeto de prédios que eram inerentemente agradáveis e
eficazes. O padrão é uma descrição do problema e da essência de sua solução, de modo que a solução possa ser
reusada em diferentes contextos. O padrão não é uma especificação detalhada. Em vez disso, você pode pensar
nele como uma descrição de conhecimento e experiência, uma solução já aprovada para um problema comum.
Uma citação do site da Hillside Group (<http://hillside.net:>), dedicado a manter informações sobre os padrões,
sintetiza seu papel no reúso:
Padrões e Linguagens de Padrões são formas de descrever as melhores práticas, bons projetos e capturar a experiência
de uma forma que torne possível a outros reusaressa experiência.
Os padrões tiveram um enorme impacto no projeto de software orientado a objetos. Além de serem soluções
já testadas para problemas comuns, tornaram-se um vocabulário para falar sobre um projeto. Você pode, portanto,
explicar seu projeto por meio de descrições dos padrões que você usou. Isso é particularmente verdadeiro para os
padrões de projeto mais conhecidos que foram originalmente descritos pela 'Gangue dos Quatro'em seu livro de
padrões (GAMMA et al., 1995). Outras descrições de padrões particularmente importantes são as publicadas em
uma série de livros de autores da Siemens, uma grande empresa europeia de tecnologia (BUSCHMANN et al., 1996;.
BUSCHMANN et al., 2007a; BUSCHMANN et al., 2007b; KIRCHER e JAIN, 2004; SCHMIDT et al., 2000).
Os padrões de projeto são normalmente associados com projeto orientado a objetos. Muitas vezes, os padrões
publicados contam com as características de objetos, como herança e polimorfismo para fornecer generalidade.
No entanto, o princípio geral de encapsular a experiência em um padrão é aquele igualmente aplicável a qualquer
tipo de projeto de software. Então, você poderia ter padrões de configuração para os sistemas COTS. Os padrões
são uma maneira de reusar o conhecimento e a experiência de outros projetistas.
Os quatro elementos essenciais dos padrões de projeto foram definidos pela 'Gangue dos Quatro', em seu livro
de padrões:
1. Um nome que seja uma referência significativa para o padrão.
2. Uma descrição da área de problema que explique quando o modelo pode ser aplicado.
3. A descrição da solução das partes da solução de projeto, seus relacionamentos e suas responsabilidades. Essa
não é uma descrição do projeto concreto; é um modelo para uma solução de projeto que pode ser instanciado
de diferentes maneiras. Costuma ser expresso graficamente e mostra os relacionamentos entre os objetos e