% ----------------------------------------------------------
\chapter{Fundamentação Teórica}\label{chp:LABEL_CHP_2}
% ----------------------------------------------------------

\section{Engenharia de Software}
A tendência é que softwares fiquem cada vez maiores e mais complexos, e isto se deve ao poder computacional crescente a cada ano. E isto faz com que os usuários criem maiores expectativas em relação aos sistemas de software, que não apenas devem transmitir informações pela internet de forma rápida e segura, mas também adaptados a necessidade de quem o usa. \cite{SOMMERVILE}.

A engenharia de software abrange os processos com um conjunto de métodos e diversas ferramentas que possibilitam aos profissionais desenvolverem sistemas de software com melhor planejamento \cite{PRESSMAN}. Tendo técnicas as quais irão se construir com base na especificação, projeto e evolução dos programas, que normalmente não são relevantes para o primeiro estágio de qualquer processo de projeto de software é o desenvolvimento de uma compreensão dos relacionamentos entre o software que está sendo projetado e seu ambiente externo. E isto é essencial para decidir qual funcionalidade o sistema terá e poderá ser apresentada ao usuário final, assim como a estruturação do próprio projeto e principalmente estabelecer os limites do sistema, ou seja, até que ponto será desenvolvido baseado no contexto do software \cite{SOMMERVILE}.

Para que isso seja planejado em uma aplicação e utilizada corretamente, de acordo com \cite[p.126]{SOMMERVILE}: \begin{quote}
    “Modelos de contexto do sistema e modelos de interação apresentam visões complementares dos relacionamentos entre um sistema e seu ambiente.”
\end{quote} 
\subsection{Caso de uso}
Segundo \citeonline{PRESSMAN}, a UML é uma linguagem padrão para descrever e documentar um projeto de software que reúne diversos grupos de notações de modelagem, sendo cada um com sua devida sintaxe e semântica predeterminadas. Cada um desses elementos pode ser extensível, permitindo, deste modo, que sejam adaptáveis às características específicas do projeto a ser desenvolvido.
 \cite{PRESSMAN}
\subsection{Processo unificado}
Processo unificado
